\documentclass[aspectratio=1610]{beamer}
\usepackage[utf8]{inputenc}
\usepackage[english]{babel}
\usepackage{amsmath}
\usepackage{enumerate}
\usepackage{graphicx}
\usepackage{verbatim}% type in mode
\usepackage{booktabs} % Allows the use of \toprule, \midrule and \bottomrule in tables
\usepackage[
backend=biber,
style=numeric,
citestyle=numeric
]{biblatex}
\usepackage{csquotes}
\addbibresource{bibfile.bib}

\graphicspath{{figures/}}%Figure Path
\setbeamertemplate{caption}[numbered] % achieve numbering of figures 
\setbeamertemplate{bibliography item}{\insertbiblabel} % achieve numbering bibliography

% \newcommand{\ipt}[1]{\input{source/#1}}
% \def\A{\mathrm{A}}
%%%Optimize picture layout
\newlength\imagewidth
\setlength\imagewidth{0.45\columnwidth}
\mode<presentation> {

% The Beamer class comes with a number of default slide themes
% which change the colors and layouts of slides. Below this is a list
% of all the themes, uncomment each in turn to see what they look like.

\usetheme{default}
%\usetheme{AnnArbor}
%\usetheme{Antibes}
%\usetheme{Bergen}
% \usetheme{Berkeley}
%\usetheme{Berlin}
%\usetheme{Boadilla}
\usetheme{CambridgeUS}
%\usetheme{Copenhagen}
%\usetheme{Darmstadt}
% \usetheme{Dresden}
%\usetheme{Frankfurt}
%\usetheme{Goettingen}
%\usetheme{Hannover}
%\usetheme{Ilmenau}
%\usetheme{JuanLesPins}
%\usetheme{Luebeck}
%\usetheme{Madrid}
%\usetheme{Malmoe}
%\usetheme{Marburg}
%\usetheme{Montpellier}
% \usetheme{PaloAlto}
%\usetheme{Pittsburgh}
%\usetheme{Rochester}
%\usetheme{Singapore}
%\usetheme{Szeged}
%\usetheme{Warsaw}

% As well as themes, the Beamer class has a number of color themes
% for any slide theme. Uncomment each of these in turn to see how it
% changes the colors of your current slide theme.

%\usecolortheme{albatross}
% \usecolortheme{beaver}
% \usecolortheme{beetle}
%\usecolortheme{crane}
% \usecolortheme{dolphin}
% \usecolortheme{dove}
%\usecolortheme{fly}
%\usecolortheme{lily}
% \usecolortheme{orchid}
\usecolortheme{rose}
% \usecolortheme{seagull}
%\usecolortheme{seahorse}
% \usecolortheme{whale}
% \usecolortheme{wolverine}

%\setbeamertemplate{footline} % To remove the footer line in all slides uncomment this line
%\setbeamertemplate{footline}[page number] % To replace the footer line in all slides with a simple slide count uncomment this line

%\setbeamertemplate{navigation symbols}{} % To remove the navigation symbols from the bottom of all slides uncomment this line
}


\title{\LaTeX\ Beamer Template}
%%% For multi Author
% \author[Zhao Chi, Ivan]
% {Zhao Chi\inst{1}\and Ivan \inst{2}}
%%% For single Author
\author{Zhao Chi}

\date[2020]
{Prepare for Game Theory Presentation, April 2020}
%%% For multi institute
% \institute{
%     \inst{1}
%     Faculty of Applied Mathematics \\
%     and Control Processes
%     \and
%     \inst{2}
%     Faculty of Applied Mathematics \\
%     and Control Processes
% }
%%% For Single institute
\institute[St.Petersburg State University]{
    St.Petersburg State University \\ Faculty of Applied Mathematics and Control Processes\\
    \textit{dandanv5@hotmail.com}
}

% \logo{\includegraphics[height=1.5cm]{CoA_Medium_color.png}\quad \includegraphics[height=1.5cm]{PMPU.jpg}}
\logo{\includegraphics[height=1.5cm]{PMPU.jpg}}

% \AtBeginSection[]
% {
%   \begin{frame}
%     \frametitle{Table of Contents}
%     \tableofcontents[currentsection]
%   \end{frame}
% }

\begin{document}
\begin{frame}
    \titlepage % Print the title page as the first slide
\end{frame}
    
\begin{frame}
  \frametitle{Table of contents slide} % Table of contents slide, comment this block out to remove it
  \tableofcontents % Throughout your presentation, if you choose to use \section{} and \subsection{} commands, these will automatically be printed on this slide as an overview of your presentation
\end{frame}

% \section{How to use this template?}
% \begin{frame}
%   \frametitle{How to use this template?}
% \end{frame}

\section{First Section}
\subsection{Block}
\begin{frame}
    \frametitle{The use of block}
    In this slide, some important text will be
    \alert{highlighted} because it's important.
    Please, don't abuse it.
    \begin{block}{Remark}
    Sample text
    \end{block}
    
    \begin{alertblock}{Important theorem}
    Sample text in red box
    \end{alertblock}
    
    \begin{examples}
    Sample text in green box. The title of the block is ``Examples".
    \end{examples}
\end{frame}

\subsection{List}
\begin{frame}
    \frametitle{The use of list}
    This frame is used to test list.
    \begin{itemize}
        \item[1.] First item in a list.
        \item[2.] Second item.
    \end{itemize}
\end{frame}

\subsection{Column}
\begin{frame}
    \frametitle{How to separate columns?}
    \begin{columns}

        \column{0.5\textwidth}
        This is a text in first column.
        $$E=mc^2$$
        \begin{itemize}
        \item First item
        \item Second item
        \end{itemize}
        
        \column{0.5\textwidth}
        This text will be in the second column
        and on a second tought this is a nice looking
        layout in some cases.
        \end{columns}
\end{frame}
\section{Section2}
\subsection{Figure}
\begin{frame}
    \frametitle{Figure}
    \begin{columns}
        \column{0.5\textwidth}
        Single figure.
        \begin{figure}[ht]
            \centering
              \includegraphics[height=1.5cm]{SPBU.png}
              \caption{SPBU}
              \label{fig:SPBU}
        \end{figure}
        \column{0.5\textwidth}
        Multi Figure.
        \begin{figure}[ht]
            \centering
            \begin{minipage}[t]{0.48\linewidth}
            \centering
            \includegraphics[height=1.5cm]{SPBU}\\
            a)
            \end{minipage} \hfill
            \begin{minipage}[t]{0.48\linewidth}
            \centering
            \includegraphics[height=1.5cm]{PMPU}\\
            b)
            \end{minipage}
            \caption{a) image ``SPBU''; b) ``PMPU''.}
            \label{fig:APairPlaintext}
        \end{figure}
    \end{columns}
\end{frame}
\subsection{Table}
\begin{frame}
\frametitle{How to use Table}
    This frame is used to show how to use Table.
    \begin{table}[ht]
        \centering
        \caption{Sample of student weight}
        \label{tab:1}
            \begin{tabular}{ccccc}
            \toprule
            Num&Gender&Age&Height/cm&Weight/kg\\
            \midrule
            1&F&14&156&42\\
            2&F&16&158&45\\
            3&M&14&162&48\\
            4&M&15&163&50\\
            \cmidrule{3-5} %添加2-4列的中线
            Average& &15&159.75&46.25\\
            \bottomrule
            \end{tabular}
        \end{table}
\end{frame}
\section{Sectoin 3}
\subsection{Multiline Equation}
\begin{frame}
    \frametitle{Multiline Equation}
    Aligning several equations with no numbers  \cite{oetiker1995not}
    .
    \begin{align*}
        x&=y           &  w &=z              &  a&=b+c\\
        2x&=-y         &  3w&=\frac{1}{2}z   &  a&=b\\
        -4 + 5x&=2+y   &  w+2&=-1+w          &  ab&=cb
    \end{align*}
    Other way.
    \begin{equation}
    \begin{aligned}
        x&=y           &  w &=z              &  a&=b+c\\
        2x&=-y         &  3w&=\frac{1}{2}z   &  a&=b\\
        -4 + 5x&=2+y   &  w+2&=-1+w          &  ab&=cb
    \end{aligned}
\end{equation}
\end{frame}
\begin{frame}
    \frametitle{Multiline Equation}
    \begin{multline}
        a + b + c + d + e + f+ g + h + i \\
        = j + k + l + m + n\\
        = o + p + q + r + s\\
        = t + u + v + x + z
    \end{multline}
    \begin{align}
        a + b + c + d + e + f+ g + h + i \\
        = j + k + l + m + n \notag\\
        = o + p + q + r + s \tag{dumb}\\
        = t + u + v + x + z
    \end{align}
\end{frame}

\subsection{Array and matrix}
\begin{frame}

Matrix
\begin{equation*}
    \mathbb{P}= 
    \begin{bmatrix}
    p_{11} & p_{12} & \ldots& p_{1n} \\
    p_{21} & p_{22} & \ldots& p_{2n} \\
    \vdots & \vdots & \ddots& \vdots \\
    p_{m1} & p_{m2} & \ldots& p_{mn}
    \end{bmatrix}
\end{equation*}

Array

\begin{equation*}
|x| =
    \begin{cases}
    -x & \text{if } x < 0,\\
    0 & \text{if } x = 0,\\
    x & \text{if } x > 0.
    \end{cases}
\end{equation*}

\begin{equation*}
    |x| =\left\{
        \begin{array}{lr}
            -x & \text{if } x < 0,\\
            0 & \text{if } x = 0,\\
            x & \text{if } x > 0.
            \end{array}\right.
    \end{equation*}
\end{frame}
% \begin{frame}[t,allowframebreaks]
%     \frametitle{References}
%     \printbibliography
% \end{frame}
\begin{frame}
    \frametitle{References}
    \printbibliography
\end{frame}
% \ipt{chapter1.tex}

% \begin{frame}
%   \cite{oetiker1995not}
% \end{frame}

\begin{frame}
    \frametitle{Thanks for your attention}
    \medskip
    \centering
    {\LARGE{\inserttitle}}\\
    \bigskip
    {\LARGE{\insertauthor}}\\
    \bigskip
    % \insertinstitute\\
    \bigskip
    \insertdate
    % {\LARGE{\centerline{Correct me if I'm wrong!}}}
    % {\centerline{Author: Zhao Chi}}
    \end{frame}
\end{document}